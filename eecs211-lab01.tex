\RequirePackage{fixltx2e}
\documentclass{tufte-handout}

\usepackage{eecs211-lab}

\title{EECS 211 Lab 1}
\author{Navigating the Unix Shell}
\date{Winter 2017}

\begin{document}

\maketitle

Today we are going over the basics of how to log into a remote computer,
use shell commands to create and edit files, and compile and run C++
code. The Northwestern EECS Machines run a Unix shell called
\progname{tcsh}.\marginnote{Using \progname{tcsh} is very similar to using
\progname{bash}, the default shell that Macs use for Terminal.app.}

The shell works as a textual conversation. It presents a prompt, like
\verb![wsc147@batman eecs211]$!. (The default EECS prompt shows the
username, the hostname, and the current working directory.) You type a
command and press enter. The shell executes the command and then prints
another prompt, waiting for further commands. For example, to list the
files in the current directory, you will run the \progname{ls} command
by typing it at the prompt:

\begin{CmdLine}
  \prompt ls
\end{CmdLine}

\marginnote[-\baselineskip]{Don't type the \prompt. That stands for your
shell prompt.}

\noindent
Before you can do that, though, we have to get you logged in.

\section{Logging in}

For the majority of you who are unfamiliar with the Unix shell, it
probably seems like a scary foreign concept reserved for computer
hackers on TV shows and movies.  However, in reality, with a little bit
of time and a few basic commands, you will realize that the Unix shell
is not something to be scared of, and in fact a very useful tool to
embrace as you continue your computer science education. Don't get
frustrated if it seems hard at first! Every great computer scientist was
at one point also unfamiliar with the shell, just like you, but with a
little bit of exposure, it will start to make sense.

SSH (secure shell) is a protocol that allows you to login remotely onto
an external system. We will be using it in order to create a connection
onto a Northwestern remote server, where we will be learning our first
Unix skills. For the first step of establishing the connection, it will
be different for Windows and Mac/Linux, but for the rest it should not
matter which OS you are on, since you'll be using the remote Unix
machine.

\subsection{Windows}

Download the SSH client PuTTY.%
\marginnote{\url{https://the.earth.li/~sgtatham/putty/latest/x86/putty-0.67-installer.msi}}
The link on the right will take you
directly to the Windows installer.
After you install PuTTY, open it up. You'll need to enter a hostname to
login to. The link on the right will take you to a list of student lab
hostnames\marginnote{\url{http://www.mccormick.northwestern.edu/eecs/documents/current-students/student-lab-hostnames.pdf}}
(such as 
\hostname{tlab-03.eecs.northwestern.edu}
or
\hostname{batman.eecs.northwestern.edu}). Ensure SSH is selected, then
press Open. You should get some sort of message asking whether or not
you trust the host. Press yes. From here, login as your EECS username
(probably the same as NetID), and your EECS password (not necessarily
your NetID password). You should now be logged into one of the
Northwestern EECS boxes!

(Note that you can configure PuTTY so that you don't have to do all of
this every time.)

\subsection{Mac/Linux}

For those of you on Mac or Linux, everything you need is already
installed. Open up your terminal\marginnote{Mac users: search
for ``terminal'' in Spotlight} and at the prompt
type a single command of the form

\begin{CmdLine}
  \prompt ssh \metavar{eecs-id}@\metavar{eecs-host}.eecs.northwestern.edu
\end{CmdLine}
\marginnote[-\baselineskip]{Don't type the \prompt.}

\noindent where \metavar{eecs-id} is your EECS username (probably your NetID)
and \metavar{eecs-host} is replaced by one of the EECS hostnames from the list
of student lab hostnames%
\marginnote{\url{http://www.mccormick.northwestern.edu/eecs/documents/current-students/student-lab-hostnames.pdf}}
(such as \hostname{tlab-03.eecs.northwestern.edu} or
\hostname{batman.eecs.northwestern.edu}).

You should get a message saying that the authenticity of the host
can't be established, and you should be asked if you want to continue
connecting.  Type ``yes'' as prompted and press Enter. Now type in your
EECS account password (not necessarily your netID password), press
Enter again, and you should be logged in remotely!

\section{Basic shell navigation}

There are a few basic commands we will be using frequently throughout
this exercise in our shell: \progname{cd}, \progname{ls}, and
\progname{pwd}, and \progname{man}.

\progname{cd} stands for ``change directory,'' and is used to change the
current directory we are looking at in our shell (our working
directory). You can think of a directory as a folder from your regular
interactions with your computer. For example the command \cmdline{cd
Documents} \marginnote{As usual, don't type the \prompt.} will look for a
directory inside our current directory called \filename{Documents}, and
if it exists, our working directory will become that
\filename{Documents} directory. If you ever want to go back to your home
directory, the command \cmdline{cd} with no argument will switch your
working directory back to your home directory. The command \cmdline{cd
..} will switch your working directory up one level from where you
currently are.

\progname{pwd} stands for ``print working directory,'' and is used to
print out the current working directory of your shell. For example, if
you have been navigating around for a while and you are lost you can
type in the command \cmdline{pwd} and you will see your directory
printed out into the shell.

\progname{ls} is short for the word ``list,'' and is used to list the contents
and subdirectories within your current working directory. You can type
the command \cmdline{ls} into your shell, and you will see all files and
directories within your current working directory.

Play around with these three commands for a few minutes in your shell,
and see what directories and files already exist on your EECS box!

\progname{man} is short for ``manual,'' and is used to access the system
manuals. For example, you can read the manual pages for \progname{pwd}
and \progname{ls} by running the commands \cmdline{man pwd} and
\cmdline{man ls}.\marginnote{Hit \keycombo{q} to quit.} To learn about
man, you can of course run \cmdline{man man}.

Once you are done playing around, type \cmdline{cd} in to navigate back
to your home directory. We will be making a new directory for this lab
using the \progname{mkdir} command.

\section{Creating new content}

\progname{mkdir} stands for make directory, and is used to create a new
directory within our current working directory. For example,
\cmdline{mkdir fun-project} will create a new directory inside our
current one called \filename{fun-project} that we can \progname{cd} into
if we so desire. We can create hierarchies of directories to keep our
files well organized.

Create a new directory inside your home directory called
\filename{lab1-dir}. Change your current working directory to
\filename{lab1-dir}, and we will now practice editing and compiling
some C++ files!

The \cmdline{emacs} command in the shell will open up the Emacs text%
\marginnote{Text editor preferences can be a fairly contentious issue
among software engineers, and if you already have experience with one
of Vim or Emacs, feel free to use whichever you already have
experience with instead of Emacs. However, for the purpose of this
class, we will be teaching using Emacs. Emacs can also seem scary at
first, but after you learn a few simple commands, it will quickly
start making sense. } editor. (On Mac/Linux, you will probably want to
use \cmdline{emacs -nw} to avoid starting the X Window System..) Pass in
a file that you want to edit (even if it hasn't been created yet), and
you can start editing that file! For example type \cmdline{emacs -nw
my\_code.cpp} and you can start editing a file called
\filename{my\_code.cpp} within your current working directory.

Inside your \filename{lab1-dir} directory create and open a file
using Emacs called \filename{animals.txt}. Note that the .cpp file
extension is what we will be using to indicate C++ files. You will see a
text editor pop up that does not look dissimilar to a
\progname{Notepad.exe} or \progname{TextEdit.app} editor from your
Windows or Mac. However, you will notice that clicking a location using
your cursor will not move your cursor to where you click :(

Inside this text editor, type in a list of your 3 favorite animals. Once
you have typed in your list, you are going to want to save your file so
you can use it later. On Emacs, saving is slightly different than other
programs. Instead of using \keycombo{Command-} or \keycombo{Ctrl-s}, you
are going to use \keycombo{Ctrl-x} followed by
\keycombo{Ctrl-s}.\marginnote{If you are curious about more Emacs
  commands, there is a nice basic list here:
\url{http://www.cs.cornell.edu/courses/cs312/2003sp/handouts/emacs.htm}.
You can also run an Emacs tutorial inside Emacs. Press \keycombo{C-h
t} -- that is, \keycombo{Ctrl-h} followed by \keycombo{t} (no
\keycombo{Ctrl}).}
(In Emacs, this is spelled ``\keycombo{C-x C-s}.'') This will save your
file to your current working directory. Now, we want to close our Emacs
window and get back to our Unix shell. In order to close our Emacs
window, we will type \keycombo{C-x C-c} (that is, \keycombo{Ctrl-x}
followed by \keycombo{Ctrl-c}).

We can ensure that our file was properly created by using the
\progname{cat} command in the shell. \progname{cat} is short for
``catenate,'' and prints out contents of a given file.  \cmdline{cat
\metavar{filename}} will print the contents of the file to the shell. If
you run \cmdline{cat animals.txt} you should see the file you just
created on your shell.

\section{Sending files back home (ET phone home)}

Now, if everything worked, we are going to want to bring our list of our
favorite animals back to our normal computers so we can access it
easier, as it is something that is very essential in our lives.  We can
do this using a program called \progname{scp}.

\progname{scp} stands for secure copy protocol, and securely copies
files to and from a remote box.  For example the command \cmdline{scp
\metavar{filename} \metavar{remote-directory}} will securely copy
\metavar{filename} to the remote directory. This also works in the
reverse direction.  And since we are trying to get our list of our
favorite animals back to our computer, we will be using that reverse
direction (\cmdline{scp \metavar{remote-file} \metavar{local-directory}}).

\subsection{Mac/Linux}

On Mac, open up a second terminal window (This can be done easily by
pressing \keycombo{Cmd-N} if you are currently on a terminal window).

Navigate to a local directory where you want to keep your
\filename{animals.txt} file using \progname{cd}, and/or create one using
\progname{mkdir}.  Now, use the \progname{scp} command to bring
the file home!  This will look something like the following:

\begin{CmdLine}
  \prompt scp \metavar{eecs-id}@\metavar{eecs-host}:lab1-dir/animals.txt .
\end{CmdLine}

\noindent
Note the second argument is a single period, which signifies the current
working directory to be the destination address, so \progname{scp} will copy the
\filename{animals.txt} from the remote machine to the current working
directory on your local, home machine. For example, my \progname{scp} command
looked something like this:

\begin{CmdLine}
  \prompt scp wsc147@batman.eecs.northwestern.edu:lab1-dir/animals.txt .
\end{CmdLine}

Look around on your local directory to make sure \filename{animals.txt}
exists; if it does you are all done!  If you aren't make sure you did
things correctly, and if you need help, raise your hand for a peer
mentor!

\subsection{Windows}

Using the search menu, search for the command prompt on your Windows
machine.  Once you have this open, you can navigate to a directory you
want to copy your \filename{animals.txt} to using the same \progname{cd}
command as before, but instead of using \progname{ls}, on Windows you
can use \progname{dir} instead. Once you have navigated to a directory
you are comfortable with (such as the \filename{Desktop} or
\filename{Documents} or what have you), we will be using the
\progname{pscp}
command that came with our PuTTY install.  This command will look
something like this:

\begin{CmdLine}
  \prompt pscp \metavar{eecs-id}@\metavar{eecs-host}:lab1-dir/animals.txt .
\end{CmdLine}

\noindent
Note the second argument is a single period, which signifies the current
working directory to be the destination address, so \progname{scp} will copy the
\filename{animals.txt} from the remote machine to the current working
directory on your local, home machine. For example, my \progname{scp} command
looked something like this:

\begin{CmdLine}
  \prompt pscp wsc147@batman.eecs.northwestern.edu:lab-dir/animals.txt .
\end{CmdLine}

Look around on your local directory to make sure \filename{animals.txt}
exists, if it does you are all done! If you aren't make sure you did
things correctly, and if you need help, raise your hand for a peer
mentor!

\section{Creating our build system}

Back on our remote logins, please navigate to your home directories
(using just the \cmdline{cd} command). Once your current working
directory is your home directory, we are going to set up a configuration
file so that you will be using the correct version of CMake, and that it
will be correct each time you remotely login to your EECS account. Don't
worry about understanding what is going on right here, it is something
that just needs to be configured for this quarter. To do this we will be
creating a file called \filename{.tcshrc} in your home directory (note
the period, and ensure this name is spelled exactly
correct).\marginnote{Remember to create this file and edit it using
Emacs we will type in \cmdline{emacs .tcshrc}.} Once this is open, type
the following line exactly into your file, save, and exit it:

\begin{verbatim}
  source /home/jesse/pub/etc/csh_profile
\end{verbatim}

Now, for this time only, type in \cmdline{exec tcsh} into the shell to
reload.

Each time you open up your remote connection (including right now) and
plan on using CMake (probably every time for first few weeks of the
class), type the \cmdline{dev} command into your shell.  This will
ensure that everyone is on the same (and correct) developer toolset.

\section{Using our build system}

As briefly mentioned in class, \progname{CMake} is our build system we
will be using for the first few weeks of the course at least.  We will
usually be giving some sort of starting structure for the projects you
will work on, and right now is no exception.  This structure is found in
a ZIP file which will need to be
downloaded onto your remote shells.  We have a command called wget which
will help download things from the internet for us.

\progname{wget} stands for ``web get,'' and downloads things from the web using the
following command format:

\begin{CmdLine}
  \prompt wget \metavar{url}
\end{CmdLine}

\noindent
For this example, we will be getting our file from this url:
\url{http://users.eecs.northwestern.edu/~jesse/course/eecs211/lab/eecs211-lab01.zip}.
So therefore, our command will be:

\begin{CmdLine}
  \prompt wget http://users.eecs.northwestern.edu/~jesse/course/eecs211/lab/eecs211-lab01.zip
\end{CmdLine}

Once we have our ZIP file, we will need to turn it from a compressed
file into a new directory, using the \progname{unzip} command.
Don't worry too much
about what this is, just understand what it does.  Use the following
command to turn this file into a directory called
\filename{eecs211-lab01}:

\begin{CmdLine}
  \prompt unzip eecs211-lab01.zip
\end{CmdLine}

Now, once we have our new directory with its files, change your
directory to \filename{eecs211-lab01} using the \progname{cd} command.  Now list
its contents using \progname{ls}, and notice that there is a
\filename{CMakeLists.txt} file, and a \filename{hello.cpp} file. The
\filename{CMakeLists.txt} file is a CMake configuration file which you
won't have to worry about too much, and the \filename{hello.cpp} file is
a C++ file which we have provided you.

When using CMake, you will want to create a subdirectory to do your
building and create your executable files the first time you set up each
project.  To do this create a subdirectory called \filename{build} using
\progname{mkdir}, and change that to be your current working directory.
Then, the first time you set up each project, you will use the
\progname{cmake} command like so: \cmdline{cmake ..}\marginnote{Note the
  two periods, signifying the directory up one level, or in this case,
the \filename{eecs211-lab01} directory.} to set up the build environment.  This should spit
out some commands into the terminal, and if everything is successful,
your CMake project is set up!

Now, you can build your program using the command \progname{make}.

The basic purpose of Make is to build your project into an executable file.
In your build directory, each time you update your code, you can run

\begin{CmdLine}
  \prompt make \metavar{target-name}
\end{CmdLine}

\noindent to create your executable called \metavar{target-name}.  In
this case, run \cmdline{make hello} to build a program called
\filename{hello}.  In your build directory, if you list its contents, you
will see a few more CMake files, but you should also now see a file
called \filename{hello}, which is your executable you can now run:

\begin{CmdLine}
  \prompt ./hello
\end{CmdLine}

\noindent
This should spit out a nice greeting.

\section{Updating our code}

So, we gave you a basic function and you were able to run it, but how do
you change the code?

Navigate back up to your \filename{eecs211-lab01} directory using cd
(\cmdline{cd ..}).  Now open up the \filename{hello.cpp} file using
Emacs, and edit it so it now says ``Aloha 211 student!'' instead of
``Hello 211 student!.'' Make sure to save and exit
Emacs.\marginnote{Remember \keycombo{C-x C-s} to save and \keycombo{C-x
C-c} to exit.}

Go back to our build directory using cd.  Try running \cmdline{./hello}
again. Did anything change?

The reason why you still see ``Hello 211 student!'' on your screen is
because while you changed your C++ code, your computer doesn't
understand the C++ code, but only the machine code you create by using
\progname{make}.  So now, run \cmdline{make hello} once again, and try
\cmdline{./hello}. Notice how you now have the correct output!  Each time
we want to change our code, we are going to need to remember to rebuild
our executable. Don't worry if you have error messages your first few
times trying to write new code, this is completely normal.  Even the
best developers in the world usually need a few tries before they can
properly build their files, so just take a deep breath, and try and
figure out what went wrong.

\section{Conclusion}

Knowing how to use the shell is an extremely important tool in computer
science. Don't worry if it is still hard for you to use, like much of
life, it is one of those things you'll just need to practice with until
it seems much more familiar!  On your own time, it would be a good idea
to continue learning more about the shell and playing around with some
more commands.\marginnote{A good resource for some basic commands is
  here:
\url{http://www.computerworld.com/article/2598082/linux/linux-linux-command-line-cheat-sheet.html}.}
Of course, come to office hours or post on Piazza with any questions or
if you want any more challenges!

\section{Useful links}

\begin{description}

  \item[Lab 1 Project]
\url{http://users.eecs.northwestern.edu/~jesse/course/eecs211/lab/eecs211-lab01.zip}

\item[EECS Lab names]
\url{http://www.mccormick.northwestern.edu/eecs/documents/current-students/student-lab-hostnames.pdf}

\item[Simple command line cheat sheet]
\url{http://www.computerworld.com/article/2598082/linux/linux-linux-command-line-cheat-sheet.html}

\item[PuTTY]
\url{http://www.putty.org/}

\item[Nice Emacs guide]
\url{http://www.cs.cornell.edu/courses/cs312/2003sp/handouts/emacs.htm}

\item[Nice Vim guide]
\url{http://www.openvim.com/}

\end{description}

\end{document}
