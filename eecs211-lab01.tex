\documentclass{tufte-handout}

\usepackage{eecs211-lab}

\title{EECS 211 Lab 1}
\author{Navigating the Unix Shell}
\date{Winter 2017}

\begin{document}

\maketitle

Today we are going over the basics of how to log into a remote computer,
use shell commands to create and edit files, and compile and run C++
code. The Northwestern EECS Machines run a Unix shell called
\progname{tcsh}.\marginnote{Using \progname{tcsh} is very similar to using
\progname{bash}, the default shell that Macs use for Terminal.app.}

The shell works as a textual conversation. It presents a prompt, like
\verb![wsc147@batman eecs211]$!. (The default EECS prompt shows the
username, the hostname, and the current working directory.) You type a
command and press enter. The shell executes the command and then prints
another prompt, waiting for further commands. For example, to list the
files in the current directory, you will run the \progname{ls} command
by typing it at the prompt:

\begin{CmdLine}
  \prompt ls
\end{CmdLine}

\marginnote[-\baselineskip]{Don't type the \prompt. That stands for your
shell prompt.}

\noindent
Before you can do that, though, we have to get you logged in.

\section{Logging in}

For the majority of you who are unfamiliar with the Unix shell, it
probably seems like a scary foreign concept reserved for computer
hackers on TV shows and movies.  However, in reality, with a little bit
of time and a few basic commands, you will realize that the Unix shell
is not something to be scared of, and in fact a very useful tool to
embrace as you continue your computer science education. Don't get
frustrated if it seems hard at first! Every great computer scientist was
at one point also unfamiliar with the shell, just like you, but with a
little bit of exposure, it will start to make sense.

SSH (secure shell) is a protocol that allows you to login remotely onto
an external system. We will be using it in order to create a connection
onto a Northwestern remote server, where we will be learning our first
Unix skills. For the first step of establishing the connection, it will
be different for Windows and Mac/Linux, but for the rest it should not
matter which OS you are on, since you'll be using the remote Unix
machine.

\subsection{Windows}

Download the SSH client PuTTY.%
\marginnote{\url{https://the.earth.li/~sgtatham/putty/latest/x86/putty-0.67-installer.msi}}
The link on the right will take you
directly to the Windows installer.
After you install PuTTY, open it up. You'll need to enter a hostname to
login to. The link on the right will take you to a list of student lab
hostnames\marginnote{\url{http://www.mccormick.northwestern.edu/eecs/documents/current-students/student-lab-hostnames.pdf}}
(such as 
\hostname{tlab-03.eecs.northwestern.edu}
or
\hostname{batman.eecs.northwestern.edu}). Ensure SSH is selected, then
press Open. You should get some sort of message asking whether or not
you trust the host. Press yes. From here, login as your EECS username
(probably the same as NetID), and your EECS password (not necessarily
your NetID password). You should now be logged into one of the
Northwestern EECS boxes!

(Note that you can configure PuTTY so that you don't have to do all of
this every time.)

\subsection{Mac/Linux}

For those of you on Mac or Linux, everything you need is already
installed. Open up your terminal\marginnote{Mac users: search
for ``terminal'' in Spotlight} and at the prompt
type a single command of the form

\begin{CmdLine}
  \prompt ssh \metavar{eecs-id}@\metavar{eecs-host}.eecs.northwestern.edu
\end{CmdLine}
\marginnote[-\baselineskip]{Don't type the \prompt.}

\noindent where \metavar{eecs-id} is your EECS username (probably your NetID)
and \metavar{eecs-host} is replaced by one of the EECS hostnames from the list
of student lab hostnames%
\marginnote{\url{http://www.mccormick.northwestern.edu/eecs/documents/current-students/student-lab-hostnames.pdf}}
(such as \hostname{tlab-03.eecs.northwestern.edu} or
\hostname{batman.eecs.northwestern.edu}).

You should get a message saying that the authenticity of the host
can't be established, and you should be asked if you want to continue
connecting.  Type ``yes'' as prompted and press Enter. Now type in your
EECS account password (not necessarily your netID password), press
Enter again, and you should be logged in remotely!

\end{document}
