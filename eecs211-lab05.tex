\RequirePackage{fixltx2e}
\documentclass{tufte-handout}

\def\ThisLabBase{eecs211-lab05}
\def\ThisLabUrl{\LabBaseUrl/\ThisLabBase.zip}

\usepackage{eecs211-lab}
\title{EECS 211 Lab 5}
\author{Game Development - Type Racer}
\date{Winter 2019}

\begin{document}

\maketitle

Today we will be looking at another a C++ program using the eecs211 game engine
in a simple example game called type racer. This program uses the Model, View,
Controller architecture described in class that allows you to define the look,
interactions, and coordinating logic between these two components. Provided are
the model file which defines the game state, the view file for drawing elements
on the screen, and a controller file for handling inputs. As well, there is a
game file that loads our dictionary and a testing file. The game is simple,
letters appear in circles on the screen and you have to type the letter you see
on the keyboard or you will lose.

\section{Game Code}

\subsection{Downloading}

For this lab we will be getting our project files from this url:
\url{\ThisLabUrl}. Extract this file into \filename{\ThisLabBase}.
Now, once we have our new directory \filename{\ThisLabBase} with its
files, we can open it up in CLion. Be careful though as CLion will
only work correctly if you open the main project directory with the
CMakeLists.txt in it. Any other directory and CLion may create a
CMakeLists.txt in it but this will not be properly configured.

\subsection{Your edits 1}

Stuff

\subsection{Your edits 2}

Stuff

\subsection{Your edits 3}

Stuff

\subsection{Testing}

Stuff

\end{document}
