\RequirePackage{fixltx2e}
\documentclass{tufte-handout}

\def\ThisLabBase{eecs211-lab05}
\def\ThisLabUrl{\LabBaseUrl/\ThisLabBase.zip}

\usepackage{eecs211-lab}
\title{EECS 211 Lab 5}
\author{Type Racer}
\date{Winter 2019}

\begin{document}

\maketitle

Today we will be looking at another a C++ program using the GE211 game engine
in a slightly more advanced example game. This game uses the Model, View,
Controller architecture described in class that allows you to define the look,
interactions, and coordinating logic between these components. Provided are
the model file which defines the game state, the view file for drawing elements
on the screen, and a controller file for handling inputs. As well, there is a
game file that loads our dictionary and runs the game. The concept of the game
is simple enough, words appear on screen as letters in circles and you have to
type the letters you see on the keyboard in order or you will lose.

\section{The Game}

\subsection{Getting the starter code}

For this lab, the starter code is provided as a ZIP file here:
\url{\ThisLabUrl}. Extract the archive file into a directory in the
location of your choosing. Once you have your new directory containing
the starter files, you can open it in CLion.

Be careful, as CLion will only work correctly if you open the \emph{main
project directory} with the \filename{CMakeLists.txt} in it. If you open
any other directory, CLion may create a \filename{CMakeLists.txt} for
you, but it won't work properly.

\subsection{General Idea}

You have been given a fully functioning type racer that loads a dictionary file
(this file can be found at \filename{Resources/dictionary.dat}), then displays
each word from the dictionary---in order---as letters inside yellow circles,
and can then take keyboard input to udpate the players progress through the
word. You progress through the word by either typing a correct letter (circle
color changes to green) or an incorrect letter (color changes to red).
Additionally, if you do not type the word in time you will finish---fail---that
word. Upon finishing a word the game loads the next until all words in the
dictionary have been exhausted. Try to identify these components and their
interactions in the code provided before continuing!

\subsection{Randomize dictionary}

In \filename{controller.cpp} there is a helper function for loading in the
dictionary file called \verb!load_dictionary()! which uses a \verb!std::vector!
to keep track of the dictionary. Your job is to implement the Fisher–Yates
shuffle algorithm and apply it to this vector before returning it so that each
game is not the same set of words in the exact same order. The algorithm
involves looping over the entire structure, choosing a random integer between 0
and the current index, and swapping the element at that random index with the
current index in your loop. Upon finishing the loop, your data will be
perfectly random.

\subsection{Score keeping}

One thing that might be nice for this game is score keeping. Implement score
keeping in this game by adding the ability to track a score in
\filename{model.cpp}. As well, you will want to see the score you have while
playing so make sure to add some score text to \filename{view.cpp} that you
can update as the player progresses.The score keeping method you should use is
5 points for every correct letter, -10 for every incorrect letter, and -30 for
every word missed due to time.

\subsection{Testing}

Unfortunately there is no reasonable way to test randomness but we can write
some tests to verify that score keeping works. Make sure to set up the game so
that it will give you the scoring condition you want, and then verify the score
changed as you expected. Add this code to \filename{game_test.cpp}.

\end{document}
