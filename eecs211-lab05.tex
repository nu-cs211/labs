\RequirePackage{fixltx2e}
\documentclass{tufte-handout}

\def\ThisLabBase{eecs211-lab05}
\def\ThisLabUrl{\LabBaseUrl/\ThisLabBase.zip}

\usepackage{eecs211-lab}
\title{EECS 211 Lab 5}
\author{Type Racer}
\date{Winter 2019}

\begin{document}

\maketitle

Today we will be looking at another a C++ program using the GE211 game
engine in a slightly more advanced example game. The concept of the game
is simple enough: Words appear on screen as letters in circles, and you
to type the letters you see on the keyboard in order.

This game uses the model--view--controller pattern not-yet-described in
class, which allows defining the look, user interaction, and ``business
logic'' of an interactive program as separate components. Provided are
the \texttt{Model} class which defines the internal game state, the
\texttt{View} class for rendering game to the screen, and the
\texttt{Controller} class for reacting to user input and tieing it all
together.

\section{Getting the starter code}

For this lab, the starter code is provided as a ZIP file here:
\url{\ThisLabUrl}. Extract the archive file into a directory in the
location of your choosing. Once you have your new directory containing
the starter files, you can open it\marginpar{
Be careful, as CLion will only work correctly if you open the \emph{main
project directory} with the \filename{CMakeLists.txt} in it. If you open
any other directory, CLion may create a \filename{CMakeLists.txt} for
you, but it won't work properly.
} in CLion.


\section{General idea}

You have been given a fully functioning Type Racer that loads a
dictionary file (which can be found at
\filename{Resources/dictionary.dat}), and then displays each word from
the dictionary---in order---as letters inside colored circles. The
player's goal is to type the word, and the controller takes keyboard
input to update the player's progress through the word. The circles
start out yellow, and as the player progresses through the word, each
circle changes to green for a correctly typed letter, or red for a
mistyped or timed-out letter. Upon finishing a word the game loads the
next word, until all words in the dictionary have been exhausted. Try to
identify these components and trace their logic in the source code
provided before continuing.

\section{Randomize the dictionary}

In \filename{controller.cpp}, one of the constructors for the
\verb^Controller^ class calls a helper function \verb!load_dictionary()!
for loading the dictionary file into a \verb!std::vector<std::string>!.
Since the dictionary file is alphabetized and the model goes through the
word vector in order, this means that you see the same words, starting
with ``a,'' every time.

Your job is to modify the code of the \verb^Model^ and \verb^Controller^
classes to randomize the order of the words after the words are read in.
You will do this with a Fisher-Yates shuffle, which is a simple and
efficient algorithm for randomly permuting the order of a vector. The
algorithm\marginpar{
      In other words, if the vector has length $n$, first you choose a
      random element from index $0$ to $n - 1$ to put in position 0.
      Then choose a random element from index $1$ to $n - 1$ to put in
      position 1, and so on.
   } is:

{
\sffamily
\begin{tabbing}
\qquad\=\+
  \textbf{procedure} $\textsc{Shuffle}(v{:}\;\textsf{vector})$: \\
\qquad\=\+
  \textbf{for} $i$ \textbf{in} $0$
        \textbf{to} $\mathrm{len}(v) - 2$: \\
\qquad\=\+
  $r \gets \textrm{a random integer from between $i$ and $\mathrm{len}(v) - 1$}$; \\
  $\mathrm{swap}(v[i], v[r])$
\end{tabbing}
}

We don't want the model to randomize the words unconditionally, because
that would make testing too difficult. So instead, the shuffling itself
should happen in a new member function of the model at the controller's
request. Here's one way you can do it:

\begin{enumerate}

  \item Add a public member function to the \verb^Model^ class whose
    purpose is to shuffle the \verb^dictionary_^ vector. For a source of
    randomness, this function should take a reference to a
    \href{https://tov.github.io/ge211/classge211_1_1_random.html}
    {\texttt{ge211::Random}}. If \verb^rng^ is a \verb^ge211::Random&^
    then you can generate a random integer between \verb^a^ and \verb^b^
    (inclusive) with the call \verb^rng.between(a, b)^.

  \item Add a call to your shuffling function to the body of the
    \texttt{Controller::\-Controller(std::string const\&)} constructor.
    That way, when the game reads words from a file their order is
    randomized, but you can also avoid the shuffling by providing the
    vector directly.

\end{enumerate}

\section{Keep score}

Another thing that would be nice for this game is to add score keeping
of some kind. You could give $2$ points for every correct letter typed,
$-5$ for every incorrect letter typed, $-1$ points for every letter
timed out, and $10$ points for every word completed without errors. When
the game ends, have it stop and display the score instead of looping on
the word ``gameover.''

Here's a plan:

\begin{enumerate}
  \item Add a private\marginpar{
    Don't write a ``setter,'' because no one needs to set the score
    from outside the model.
    } member variable to the model to hold the score,
    and a public member functiom to allow the view to access it.

  \item Figure out how to detect the scoring events in the model code,
    and update\marginpar{
      Define constants for the event values. No magic numbers!
    } the score for each.

  \item Add private
    \href{https://tov.github.io/ge211/classge211_1_1_font.html}{\texttt{ge211::Font}}
    and
    \href{https://tov.github.io/ge211/classge211_1_1sprites_1_1_text__sprite.html}{\texttt{ge211::Text_sprite}}
    member variables to the view class. For creating the font, note that
    \texttt{"sans.ttf"} is included with GE211. For the text sprite, the
    initial text should be the number 0.

  \item Modify \texttt{View::draw} to:
    \begin{enumerate}
      \item
        \href{https://tov.github.io/ge211/classge211_1_1sprites_1_1_text__sprite.html\#afe7c024ae674fec2431b9d3eb9ad4173}{reconfigure}
        the text sprite to contain the current score, and
      \item add the score sprite to the sprite set.
    \end{enumerate}
    In order\marginpar{
      You might also modify the view to keep track of the last score that it
      saw, so that it only needs to reconfigure the text sprite when the score
      changes.
    } to reconfigure the text sprite, you will need to create
    a
    \href{https://tov.github.io/ge211/classge211_1_1sprites_1_1_text__sprite_1_1_builder.html}{\texttt{ge211::Text_sprite::Builder}}
    and then add text to it. It looks something like this:
\begin{Code}
my_text_sprite_.reconfigure(
    ge211::Text_sprite::Builder(my_font_) << my_value)
\end{Code}
\end{enumerate}

\section{Testing}

It's hard testing whether your shuffle is producing permutations
uniformly. You could do repeated trials and check that you get a
reasonable distribution, but that's fairly tricky as soon as $n > 2$.
Easier, however, is to use
\href{https://en.cppreference.com/w/cpp/algorithm/is_permutation}
{\texttt{std::is_permutation}} to check that the result of your shuffle
is a permutation of the original. \texttt{std::is_permutation} can take
two begin--end pairs of \emph{iterators} to compare, and
\texttt{std::vector} has public member functions \texttt{begin()} and
\texttt{end()} for getting such iterator pairs. Thus, you can check
whether two vectors \texttt{v} and \texttt{w} are permutations of each
other with:

\begin{Code}
    CHECK( is_permutation(v.begin(), v.end(),
                          w.begin(), w.end()) );
\end{Code}

You should also test that your model detects scoring events and keeps
score properly.

\section{Other ideas}

What if the game displayed a count-down timer for each letter's timeout?
(How can you format seconds with one decimal digit
\href{https://en.cppreference.com/w/cpp/io/manip/setprecision}{using
C++'s iotreams}?) Or instead of the numeric time, what if it showed a
bar whose length shrunk as the time ran out? (Does that require creating
a new \texttt{Rectangle_sprite} to change its size, or can you scale it
by passing a
\href{https://tov.github.io/ge211/classge211_1_1geometry_1_1_transform.html}
{ge211::Transform} to
\href{https://tov.github.io/ge211/classge211_1_1_sprite__set.html\#a567a6cc041710e43a2511234590cc8b9}
{the four-argument form of \texttt{add_sprite}}?)

Can you make the game time the entire word for 
\texttt{2 * current_word_.size()} seconds
instead of 2 seconds per letter as it does
now? (Then you'd really want some kind of display of the time
remaining.)

\end{document}
