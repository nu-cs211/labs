\documentclass{tufte-handout}

\usepackage{211lab}
\LabInfo{7}{Bejeweled}

\begin{document}

\maketitle

We will be looking at yet another a C++ program using the GE211 game
engine in a reasonably advanced example game. You may have played this game
in various forms such as in Candy Crush or other tile swapping games, but the
basic concept is to destroy sets of like tiles by swapping two tiles to
create a set. In this version of the game a set will be considered 4 or more
tiles of the same type.

This game uses the model--view--controller pattern not-yet-described in
class, which allows defining the look, user interaction, and ``business
logic'' of an interactive program as separate components. Provided are
the \texttt{Model} class which defines the internal game state, the
\texttt{View} class for rendering game to the screen, and the
\texttt{Controller} class for reacting to user input and tying it all
together. In addition, because the tiles in this game are rather complex
themselves, we have \filename{tile.hxx} which sets up
\texttt{Board::position},  an \texttt{Tile::apply_action} function to
apply an action which is presumed to be a subclass of
\texttt{Tile::Action} to the tile. (a different subclass of
\texttt{Tile::Action} can provide different actions for special types of
tiles), a \texttt{Tile::symbol} to represent the text inside the tile
(for special types in this case. We can also use another representation
strings for normal tiles). All this allows creation of more tile
handlers which can each have special destructive powers. You can see
that we have provided the normal action handler (which returns an empty
set, meaning that for the specific action, we just want to follow the
normal Bejeweled rules), and a horizontal lazer (deletes all tiles in
the row in addition to the set we created by swapping), inside
\filename{actions.cxx}.

\CxxPrelims

\section{General idea}

The version of Bejeweled that you have been given a board inside \filename{bejeweled.cxx} (defaults to 10 by 8) with
several (defaults to 6) groups - which will behave as tile colors for
grouping same colors - and as many types as you have tile handlers (starting
from 2). From here, the controller decides when to update a frame and in each
update \verb!model_.step()! is called which is the brains behind
finding what needs to change, detecting the set of tiles to be destroyed
(including any caused by destroying a special type), and removes them as
needed. Looking through this function and the other functions it uses in
the \verb^Model^ class should help you understand how the game is utilizing
tiles. The \verb^Controller^ also utilizes some of the \verb^Model^ functions
in \verb!Controller::on_mouse_up! ,which (when the view isn't going through
animations) on first click of a valid tile will select that tile and on
second click of another tile will attempt to swap them. Upon swapping and
creating a set to be destroyed, that set of tiles will be removed and the
tiles above them will be shifted down (\texttt{Model::falling_step}), and new random tiles will also be
shifted down to fill the gaps created at the top. All of these changes are
animated by the \verb^Controller^ class (which will make the program unresponsive to
input while it displays the changes slow enough for you to actually see what
happens. See the implementation of \texttt{Controller::can_animate())}.

\section{More Valid Swaps}

Let's put more excitement in our game by adding new features. 

\begin{enumerate}
  \item Change \filename{actions.cxx}  and the main class (\filename{bejeweled.cxx}); and add a new tile with special new features. For the lab, you can just follow the logic behind adding horizontal lazer type of action.

  \item Add a sprite at the top left corner to keep the score. Refer to the previous lab to figure out how.
\end{enumerate}

\end{document}
