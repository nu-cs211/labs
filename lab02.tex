\documentclass{tufte-handout}

\usepackage{211lab}

\LabInfo{2}{Control Statements, Functions and Structures}

\begin{document}

\maketitle

Today we are going to practice navigating in the shell and writing basic
C code. 

\section{Getting Started}

Let's get started by logging into a remote Northwestern workstation. We
did this last week, but if you need help remembering the steps, they are
included below.\marginnote{The list of remote Northwestern workstations
can be found here: \LabHostnamesLink}

\section{Enabling the CS 211 build environment}

Type the command \cmdline{exec fish} into the shell to execute
\progname{fish} in place of your current shell. (If your shell says that
\progname{fish} isn’t found, that likely means you missed the step in
Lab 1 where you were supposed to run \cmdline{~jesse/setup211}.)

\section{Getting the code}

Recall our basic Unix commands: \progname{cd}, \progname{ls},
\progname{mkdir}, and \progname{pwd}. What do they stand for and what do
they do? If you don't remember, try reading\marginnote{Run \cmdline{man
ls} or \cmdline{man pwd}.} their manual pages.

We suggest that organize your home directory by keeping your CS 211
files in subdirectory named \filename{cs211/},\marginnote{Note that the
  directories \filename{cs211/} and \filename{\plaintilde cs211/} do not
  mean the same thing. The former means a subdirectory of the current
  directory named ``cs$211$,'' whereas the latter means the home directory
  for a user named ``cs$211$.'' When \filename{\plaintilde}
  is written by itself, it means the \emph{current user’s} home
  directory. Given all that, what do you think
\filename{\plaintilde/cs211/} means?} but it's up to you. If you have
such a directory, change into it and then extract the tarball for this
lab:

\begin{CmdLine*}
  \C cd cs211\\
  \C tar -xkvf \plaintilde cs211/lab/\ThisLabBase.tgz\\
\end{CmdLine*}

\noindent
You should now have a directory called
\filename{\ThisLabBase}.

\section{Writing the code}
Navigate into to your \filename{\ThisLabBase} directory using
\progname{cd}, and open up \filename{src/lab.c} in Emacs using

\begin{CmdLine*}
  \C emacs src/lab.c\\
\end{CmdLine*}

\noindent
Notice that there is already some skeletons of functions and some code
in \functionref{main}{} here.

\subsection{Iteration}

First, find the function called
\functionref{sum\_numbers}{}.\marginnote{Notice that
\functionref{sum\_numbers}{} is going to return an {\codestyleKeyword
int}.} We are going to use this function to sum up all of the numbers
from 1 to \varname{num}.  If you remember from class, we have a few ways
of iterating through numbers, most notably ^for^ and ^while^.  We will
be using both, but first we will be using ^while^.

\subsection{{\codestyleKeyword while} loops} As we learned in class, a  ^while^
loop has the following syntax:\marginnote{Note that in while loops we usually
will use a Boolean expression for \metavar{test-expression} (an expression
which evaluates to `true` or `false`)}

\begin{Code}
    while (‹\metavar{test-expression}›) {
        // Repeats the body statements until the test
        // expression evaluates to `false`:
        ‹\metavar{body-statements}›
    }
\end{Code}


Use a while loop inside our \functionref{sum\_numbers}{} in order to
add the numbers from 1 to \varname{num} together.\marginnote{Remember
that we have the ^++^ and ^+=^ operators if we need them.} Make sure to
use a ^return^ statement to return the sum that we aggregated!

Once you think that your function works as intended, save and exit
emacs\marginnote{\keycombo{C-x C-s} to save and \keycombo{C-x C-c} to
exit}. If you remember from last week, we used the \filename{make}
command in order to turn our C file into machine code. Run:

\begin{CmdLine*}
  \C make lab\\
\end{CmdLine*}
\marginnote{Remember, make works as follows: \cmdline{make [target]}. Target is the name of the executable file that will be built by the make command.}

If everything works, if we list our files, we should now see a file
called \filename{lab}. To run it, the command is:

\begin{CmdLine*}
  \C ./lab\\
\end{CmdLine*}

See if your value looks right!  If it doesn't, don't worry, Rome wasn't
built in a day.  Try and try and see what went wrong.\marginnote{Error
messages may look scary, but in reality, they're there to help you!  Not
intimidate you!}  Play around with the value of \varname{num} and see
how it affects the result.

\subsection{{\codestyleKeyword for} loops}

Once we have everything working with our ^while^ loop, let's work on
using a ^for^ loop.  To help you remember the syntax, here is an example of a for loop to print out the values from 1 to 5

\begin{Code}
    for (int i = 1; i <= 5; ++i) {
        // Note that the code inside the curly braces is the code
        // that is executed for each iteration of the `for` loop.
        printf("%d\n", i);
    }
\end{Code}

Go back to our \functionref{sum\_numbers}{} function, and try to replace
your ^while^ loop expression with a ^for^ loop to accomplish the same task.

Once you are done, make and run your file.  See if everything looks the same!  If not, no worries, go back and try again!


\subsection{Structs}


Structs are an important tool in C for grouping data together.  In your
\filename{lab.c} you will notice that we created an ^apple^ structure
for you.\marginnote{How 'bout 'dem apples?}  This is so that we can
organize attributes of ^struct apple^s together in a convenient way.  If
you look inside our ^apple^ struct, we decided that we will want to know
the weight, the variety, and the color of our apples.  In our
\functionref{main}{}, we created an example of a Red Delicious apple.
Now, create your own type of apple (you'll need to define it as a type
^struct apple^), and give it those three attributes.  Add in a print
statement (we gave you an example one) to print out your new ^struct
apple^.  Make and run your \progname{lab} program!  Hopefully everything
works as expected!  If not, don't fret or get upset, go back and make
changes!

Now that we know all about ^struct apple^s, let's create our own
structure. Define a structure of your favorite animal, and give your
animal three attributes, with one of them being age. Don't forget to
give your attributes types. You can create a new struct that looks very
similar to the ^struct apple^ we created.\marginnote{Don't forget the
semicolon after the closing brace.}

Once you have created your animal, go into \functionref{main}{} and
create an instance of your animal, assign it those three attributes, and
then create a print statement to print out information about your
animal.  These print statements are getting annoying; we'll tackle that
soon.

Make and run \progname{lab}, and see if your new animal shows up the way that
you intended.  Hopefully everything works!  If not, as usual, go back
and try and find what went wrong and update your code. 


\subsection{Creating your own function}

So far, we've been filling in skeleton functions that were provided for
you. Now it's time to write your own function from scratch.  Remember
how annoying it was to type out the ^printf^ lines each
time you wanted to print out your animal?  We're going to abstract that
out and replace it with a simple call to a function! 

Write a function called \functionref{print\_animal}{} that uses
\functionref{printf}3\marginnote{
  The ``\functionref{}3'' after ``\functionname{printf}'' gives the
  section in the Unix manual that contains the documentation for the
  \functionref{printf}{} function, which means you can look it up with
  the command \cmdline{man 3 printf}. If you omit the section number 3
  and run just \cmdline{man printf}, you get documentation for the
  shell's ^printf^ command rather than the C library’s
  \functionref{printf}{} function.
} to print out your animal's three attributes.
Note that this should take in one argument(of the same type as your
animal struct).  Think about what type your function should
be!\marginnote{Note that the \varname{void} return type signifies that
nothing is returned.}

Once you wrote your function, go to your \functionref{main}{} function
and replace your print statement with a call to
\functionref{print\_animal}{}.\marginnote{remember to pass your animal
instance to the function.}

Make and run \progname{lab}, and see if everything still works!  

\subsection{Control statements}

Now that we have gotten the hang of structs and functions, let's
practice our control statements.  Go back to our
\functionref{print\_animal}{} function.  Remember from HW1 and class that
^if^ statements have the following basic syntax:

\begin{Code}
    if (‹\metavar{test-expression}›) {
        // Do these if the test expression is `true`:
        ‹\metavar{then-statements}›
    } else {
        // Do these if the test expression is `false`:
        ‹\metavar{else-statements}›
        // (The `else` clause is optional.)
    }
\end{Code}

Using an ^if^--^else^ statement, check your animals age and add to your
\functionref{print\_animal}{} function a line to print out ``This animal is
old!'' if the animal is at least 10 and print out ``This animal is not
that old'' if the animal is younger than 10.   

Make and run \progname{lab} and see if this feature is working!

\end{document}
