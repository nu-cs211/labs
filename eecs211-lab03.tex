\RequirePackage{fixltx2e}
\documentclass{tufte-handout}

\def\ThisLabBase{eecs211-lab03}
\def\ThisLabUrl{\LabBaseUrl/\ThisLabBase.tgz}

\usepackage{eecs211-lab}
\title{EECS 211 Lab 3}
\author{Strings}
\date{Winter 2019}

\begin{document}

\maketitle

Today we are going to practice manipulating "strings".

\section{Getting Started}
Let's get started by logging into a remote Northwestern server. We did this last week, but if you need help remembering the steps, they are included below.\marginnote{The list of remote Northwestern servers can be found here: \url{http://www.mccormick.northwestern.edu/eecs/documents/current-students/student-lab-hostnames.pdf}}

\subsection{Windows}
Open PuTTY. You'll need to enter a hostname to login to. The link on the right will take you to a list of student lab hostnames (such as  \hostname{tlab-03.eecs.northwestern.edu} or \hostname{batman.eecs.northwestern.edu}).  Ensure SSH is selected, then press Open. When prompted, enter your EECS username and password (not necessarily the same as your NetID password) and you're good to go.

\subsection{Mac/Linux}
Open up your terminal. At the prompt, use the ssh command of the form
\begin{CmdLine}
  \prompt ssh \metavar{eecs-id}@\metavar{eecs-host}.eecs.northwestern.edu
\end{CmdLine}
\noindent where \metavar{eecs-id} is your EECS username (probably your NetID)
and \metavar{eecs-host} is replaced by one of the EECS hostnames from the list
of student lab hostnames (such as \hostname{tlab-03.eecs.northwestern.edu} or
\hostname{batman.eecs.northwestern.edu}).
When prompted, type in your EECS username and password (not necessarily your NetID password), press
Enter again, and you should be logged in remotely!

\section{Getting the code} Recall our basic Unix commands:
\progname{cd}, \progname{ls}, \progname{mkdir}, and \progname{pwd}. What
do they stand for and what do they do? Ask your TA if you don't
remember.\marginnote{Or ask Google.} Use the following
\progname{curl}-and-\progname{tar} pipeline to download and extract the
code into your directory of choice. We suggest that you keep your EECS
211 files in an \filename{eecs211/} subdirectory of your home directory,
but it's up to you.

\begin{CmdLine}
  \prompt curl \$URL211/lab/\ThisLabBase.tgz | tar zxv
\end{CmdLine}

You should now have a directory called \filename{\ThisLabBase}.

\subsection{Setting up the build system}
Type the \cmdline{dev} command into the shell to ensure that you are
using the correct developer toolset. You must do this every time you
open a remote connection and plan on compiling C code.

\section{Writing the code}
Navigate to your \filename{\LabBaseUrl} directory, and open up
\filename{src/lab2.c} in Emacs using
\begin{CmdLine}
  \prompt emacs -nw src/lab3.c
\end{CmdLine}
Notice that there is already some skeletons of functions and some code in \functionname{main()} here.

\subsection{str\_chr}
First, find the function called
\functionname{str\_chr}.\marginnote{Notice that \functionname{str\_chr} is going to return an char *.}We are going to use this function to determine if the character \varname{c} exists in the string \varname{s}. If you remember from class, we have a few ways of iterating, most notably \functionname{while} which is what you will use for this function.

\subsection{While loops}
As we learned in class, a \functionname{while} loop has the following
syntax:\marginnote{Note that in while loops we usually will use a boolean expression for <expr> (an expression which returns True or False)}

\begin{Code}
    while (<expr>) {
        // Looping through code here
        // Until <expr> is false
    }
\end{Code}

Use a while loop inside our \functionname{str\_chr} in order to see if \varname{c} is every equal to any one of the charcters in \varname{s}.\marginnote{Remember that we have the \functionname{++} function to help us.}
Make sure to use a \functionname{return} statement to return the char * that is left from our search (or a NULL if nothing is found)!

Once you think that your function works as intended, save and exit emacs\marginnote{\keycombo{C-x C-s} to save and \keycombo{C-x C-c} to exit}. If you remember from last week, we used the \functionname{make} command in order to turn our C++ file into machine code. Run:
\begin{CmdLine}
  \prompt make build/lab3
\end{CmdLine}
\marginnote{Remember, make works as follows: \cmdline{make [target]}. Target is the name of the executable file that will be built by the make command.}
If everything works, if we list our files, we should now see a file called \filename{lab3}.  Enter the command
\begin{CmdLine}
  \prompt build/lab3
\end{CmdLine}
See if your value looks right!  If it doesn't, don't worry, Rome wasn't built in a day. Try and see what went wrong.\marginnote{Error messages may look scary, but in reality, they're there to help you!  Not intimidate you!}  Play around with the value of \varname{s} and \varname{c} to see how it affects the result.

\subsection{is\_prefix\_of}

Once we have everything working with our \functionname{str\_chr}, let's
move on to a similar function called
\functionname{is\_prefix\_of}\marginnote{Notice that
\functionname{is\_prefix\_of} is going to return an bool.}. This
function is similar to \functionname{str\_chr} in that it loops through a string to find something, but the difference here is that we are looking for a substring - not just a character. Since both of the inputs are "strings" (char *), you will need to check that not only one character matches in the substring (\varname{needle}), but that every character matches. Return true if the \varname{needle} is fully contained by the \varname{haystack}.

Once you are done, make and run your file. See if your function properly identifies prefixes. If not, no worries, go back and try again!

\subsection{str\_str}

Once the function \functionname{is\_prefix\_of} is working, write a new
function \functionname{str\_str} that uses \functionname{is\_prefix\_of} to determine if a word exists anywhere in another word. To check if the search word (\varname{needle}) is in the \varname{haystack}, first check to see if it is a prefix of \varname{haystack}. If \varname{needle} is not the prefix to \varname{haystack}, try to see if \varname{needle} is the prefix of everything but the first letter of \varname{haystack}. This loop will effectively check for the subword \varname{needle} in every possible position inside \varname{haystack}. Make sure to return \varname{haystack} if you find the subword and NULL if you don't.

Make and run lab3, and see if \functionname{str\_str} works the way that you intended.  Hopefully everything works!  If not, as usual, go back and try and find what went wrong and update your code.

\subsection{interpolate}

Now using what we have learned about how to manipulate strings we are
going to write our own version of \functionname{printf} called
\functionname{interpolate}. Interpolate will return a char *, and takes
as input a const char *, a const char * array, and a final char *. The
first input (const char *) \varname{format} will contain our format
string. This string is what our program will work through to try and
come up with an output string. The second input is \varname{args} and
this holds the elements that we will be placing into the new string. The
final input is simply our \varname{buffer}, where we will build
everything to return. The rules for our format string are going to be
that you will fill in text any time you see \{\}. So, a string that
looks like "Hello \{\}!" with and argument of "Jason" would return
"Hello Jason". As well, we want to allow our format strings to have
modifiers. If you just see \{\}, then return exactly what you got as
input but if you see \{\^\} then make the input uppercase and if you see
\{v\} then make the input lowercase. This is a complex problem so it might be useful to break it down into the component parts: identifying format string and modifiers, filling the string in (modified).

Once you have this done, make and run lab3, and see if this feature is working!

\end{document}
