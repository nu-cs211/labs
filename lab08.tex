\documentclass{tufte-handout}

\usepackage{211lab}
\LabInfo{8}{Asteroids}

\begin{document}

\maketitle

Today is an exercise in software archaeology. You will be given a
complete program---your goal is to understand it well enough to modify
and improve it.

\CxxPrelims

\section{Getting oriented}

Your first step in dealing with a new codebase is to gain a basic
understanding of the structure and how it fits together. In
particular, you need to understand what significant types (classes,
structs, or enum classes) are defined by the code, and how they're
related. Are any derived classes of any others (the \emph{is-a}
relation)? Do any contain instances of or references to instances of any
other (the \emph{has-a} relation)? Of particular interest in this
assignment is the inheritance hierarchy, and how it produces the
behaviors of different kinds of objects.

\section{Possible goals}

Here are some ways you could try modifying the program:

\begin{itemize}

  \item Instead of ending the game after dying once, keep a life count and
    give the player multiple chances.

   \item Add a game-over screen that allows the player to try again.

   \item Make \marginpar{Like classic Asteroids.} objects wrap around at
     the screen edge instead of colliding with it.

   \item Add \marginpar{Power-ups should spawn randomly and disappear when
     acquired.} power-ups, such as an extra life or faster spaceship.

   \item Rate-limit the firing mechanism to allow $n$ shots before
     requiring an $r$ second reloading delay.

\end{itemize}

\end{document}
