\documentclass{tufte-handout}

\usepackage{211lab}

\LabInfo{3}{Strings}

\begin{document}

\maketitle

Today we are going to practice manipulating \emph{strings.}

\section{Getting Started}

The starter code is available at \filename{\ThisLabTgz}, so you can
extract it into your current working directory with the command

\begin{CmdLine*}
  \C tar -xkvf \ThisLabTgz\\
\end{CmdLine*}

\noindent Your current working directory should now contain a
subdirectory called \filename{\ThisLabBase}.

\section{Writing the code}

Navigate to your \filename{\ThisLabBase} directory and open up
\filename{src/lab3.c} in Emacs:

\begin{CmdLine*}
  \C emacs src/lab3.c \\
\end{CmdLine*}

\noindent
Notice that there is already some skeletons of functions and some code in \functionname{main()} here.

Now open \filename{src/lab3\_funs.c}. Note that comments explaining
functionality are placed in \filename{src/lab3_funs.h}.

\subsection{{const char *str\_chr(const char *s, char c)}}
First, find the function called
\functionname{str\_chr}.\marginnote{Notice that \functionname{str\_chr}
is going to return a \verb^const char *^.}
We are going to use this function to determine if the character
\varname{c} exists in the string \varname{s}, and if so, where. If you
remember from class, we have a few ways of iterating, most notably
\texttt{while} which is what you will use for this function.

\subsection{While loops}
As we learned in class, a \texttt{while} loop has the following
syntax:\marginnote{Note that in while loops we usually will use a
boolean expression for <expr> (an expression which returns \texttt{true} or
\texttt{false}.)}

\begin{Code}
    while (<expr>) {
        // Looping through code here
        // Until <expr> is false
    }
\end{Code}

Use a while loop inside our \functionname{str\_chr} in order to see if
\varname{c} is ever equal to any one of the charcters in
\varname{s}.\marginnote{Remember that we have the \texttt{++} operator to help us.}
Make sure to use a \texttt{return} statement to return the
\verb!char *! if we find it (or a \texttt{NULL} if nothing is found). The returned
\verb!char *! should point to the first instance of the character in the string.

Once you think that your function works as intended, save and and try
compiling and running it.\marginnote{\keycombo{C-x C-s} to save} If you
remember from last week, we used the \functionname{make} command in
order to turn our C file into machine code. Run:
\begin{CmdLine*}
  \C make lab3\\
\end{CmdLine*}
\marginnote{Remember, make works as follows: \cmdline{make [target]}.
Target is usually the name of the executable file that will be built by the make command.}
If everything works, if we list the files in \texttt{.}, we should now see a file called \filename{lab3}.  Enter the command
\begin{CmdLine*}
  \C ./lab3\\
\end{CmdLine*}
See if your value looks right!  If it doesn't, don't worry, Rome wasn't built in a day. Try and see what went wrong.\marginnote{Error messages may look scary, but in reality, they're there to help you!  Not intimidate you!}  Play around with the value of \varname{s} and \varname{c} to see how it affects the result.

\subsection{{bool is\_prefix\_of(const char *haystack,
      const char *needle)}}

Once we have everything working with our \functionname{str\_chr}, let's
move on to a similar function called
\functionname{is\_prefix\_of}\marginnote{Notice that
\functionname{is\_prefix\_of} is going to return a \texttt{bool}.}. This
function is similar to \functionname{str\_chr} in that it loops through
a string to find something, but the difference here is that we are looking for
a substring - not just a character. Also, the substring needs to be positioned
at the start of the test string. Since both of the inputs are ``strings''
(\verb!char *!), you will need to check that not only one character matches in
the substring (\varname{needle}), but that every character matches. Return true
if the first characters of \varname{haystack} entirely match \varname{needle}.

Once you are done, make and run your file. See if your function properly identifies prefixes. If not, no worries, go back and try again!

\subsection{{const char *str\_str(const char* haystack, const
char *needle)}}

Once the function \functionname{is\_prefix\_of} is working, write a new
function \functionname{str\_str} that uses \functionname{is\_prefix\_of} to
determine if a word exists anywhere in another word. To check if the search
word (\varname{needle}) is in the \varname{haystack}, first check to see if it
is a prefix of \varname{haystack}. If \varname{needle} is not the prefix to
\varname{haystack}, try to see if \varname{needle} is the prefix of everything
but the first letter of \varname{haystack}. This loop will effectively check
for the subword \varname{needle} in every possible position inside
\varname{haystack}. Return a pointer to the start of the first instance of
\varname{needle} in \varname{haystack} if you find it, and NULL if you don't.

Make and run lab3, and see if \functionname{str\_str} works the way that you intended.  Hopefully everything works!  If not, as usual, go back and try and find what went wrong and update your code.

%% BRG: This was way too much work for a lab and duplicates what students are going
%% to learn in the homework anyways. I'm cutting it.

%\subsection{{char* interpolate(const char *format, const char *args[], char *buffer)}}
%
%Now using what we have learned about how to manipulate strings we are
%going to write our own version of \functionname{sprintf}(3) (a relative
%of \functionname{printf}(3)) called
%\functionname{interpolate}. Interpolate will return a \verb!char *!, and takes
%as input a \verb!const char *!, an array of \verb!const char *!, and
%a final \verb!char *!. The
%first input (\verb!const char *!) \varname{format} will contain our format
%string. This string is what our program will work through to try and
%come up with an output string. The second input is \varname{args} and
%this holds the elements that we will be placing into the new string. The
%final input is simply our \varname{buffer}, where we will build
%everything to return. The rules for our format string are going to be
%that you will fill in text any time you see \verb!{}!. So, a string that
%looks like \verb^"Hello {}!!"^ with and argument of \texttt{"Jason"} would return
%\verb!"Hello Jason"!. As well, we want to allow our format strings to have
%modifiers. If you just see \verb^{}^, then return exactly what you got as
%input but if you see \verb!{^}! then make the input uppercase and if you see
%\texttt{\{v\}} then make the input lowercase. This is a complex problem so it might be useful to break it down into the component parts: identifying format string and modifiers, filling the string in (modified).
%
%Once you have this done, make and run lab3, and see if this feature is working!

\end{document}
