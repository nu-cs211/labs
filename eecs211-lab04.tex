\RequirePackage{fixltx2e}
\documentclass{tufte-handout}

% Be sure to check and update this:
\def\cLionurl{https://www.jetbrains.com/clion/download}
\def\macSDLurl{https://users.eecs.northwestern.edu/~jesse/course/eecs211/files/SDL2-all.dmg}
\def\MinGWSDLurl{https://users.eecs.northwestern.edu/~jesse/course/eecs211/files/MinGW-SDL2.exe}
\def\appleclangurl{https://developer.apple.com/downloads/}
\def\jetbrainsAccounturl{https://www.jetbrains.com/shop/eform/students}
\def\ThisLabBase{eecs211-lab04}
\def\ThisLabUrl{\LabBaseUrl/\ThisLabBase.tgz}

\usepackage{eecs211-lab}
\title{EECS 211 Lab 4}
\author{Game Development}
\date{Winter 2019}

\begin{document}

\maketitle

Today we will be looking at a C++ program using the eecs211 game engine in a
simple example game. Provided are the model file which defines the game state
and interactions, and a game file which uses the model to create our game and
draw the necessary shapes in the correct screen locations. The game is simple,
you control two circles on the screen - one with your mouse and one with your
keyboard - when the two circles overlap, they change color. Before we can get
started we need to install our development environment. You\textquotesingle ll
need a C++ compiler, the CLion IDE, and the SDL2 graphics libraries.

\section{Downloads}

For all platforms, you will need the \hyperlink{\cLionurl}{CLion installer}%
\marginnote{\url{\cLionurl}}.

Additional installation varies by platform.

\subsection{Mac}

You will need to download our custom \hyperlink{\macSDLurl}{SDL2 disk image}%
\marginnote{\url{\macSDLurl}}.

\subsection{Windows}

You will need to download our custom installer for
\hyperlink{\MinGWSDLurl}{MinGW-w64 with SDL2}%
\marginnote{\url{\MinGWSDLurl}}.

\subsection{Linux}

Make sure you have a working C++14 toolchain installed. You should
also install the development packages for SDL2, SDL2\_image, SDL2\_ttf,
and SDL2\_mixer.

\section{Setup}

\subsection{Mac}

Mac OS automatically installs its toolchain when you attempt to use it
from the command line for the first time; you will still have to
install the SDL2 libraries yourself.

1. Thus, to install developer tools, run the Terminal program (from
Applications/Utilities) to get a command prompt. At the prompt, type
\begin{verbatim}
  clang
\end{verbatim}
and press return. If it prints \cmdline{clang: error: no input files}
then you have it installed already. Otherwise, a dialog box will pop
up and offer to install the command-line developer tools for you. Say yes.

(Alternatively, you can install the latest version of Command Line Tools
for OS X manually from \hyperlink{\appleclangurl}{Apple}%
\marginnote{\url{\appleclangurl}}, or install XCode from the App Store.)

2. Once you have the developer tools installed, you need to install
the SDL2 libraries. Open the SDL2-all.dmg disk image and drag the
four frameworks into the linked /Library/Frameworks directory. You may
have to authenticate as an administrator.

\subsection{Windows}

On Windows, you need to install MinGW-w64 (the C++ compiler):

1. Run the MinGW-SDL2.exe installer and follow the prompts to install
MinGW-w64. You should usually install it to C:\textbackslash MinGW, but wherever
you install it, take note, as you will have to configure CLion to find
it later.

\subsection{All}

On all platforms you will need to follow these steps to set up the CLion IDE:

1. Register for a student JetBrains account on
\hyperlink{\jetbrainsAccounturl}{their website}%
\marginnote{\url{\jetbrainsAccounturl}}.

2. Follow the instructions in your email to activate your account.

3. Run the CLion installer-defaults should be fine. (Windows: Check all
of the ''Create associations'' boxes when they appear.)

Windows only: Set the toolchain in CLion to the location where you
installed MinGW. The folder you select should contain subfolders with
names like bin and lib. Ignore any warnings about version numbers.

\section{Game Code}

\subsection{Downloading}

For this lab we will be getting our project files from this url:
\url{\ThisLabUrl}. You can use the wget and tar commands (below) to extract this
folder or you can download and open the zip manually.

\begin{CmdLine}
  \prompt wget \$URL211/\ThisLabBase.tgz
  \prompt tar xvf \ThisLabBase.tgz
\end{CmdLine}

Now, once we have our new directory \filename{\ThisLabBase} with its files,
we can open it up in CLion. Be careful though as CLion will only work correctly
if you open the main project directory with the CMakeLists.txt in it. Any other
directory and CLion may create a CMakeLists.txt in it but this will not be
properly configured.

\subsection{Inverted Arrows}

Currently, there is an error with this code - if you run the program and try
to control the circle with your arrow keys you will move in the opposite
direction of what you intended. Locate the bug for this and fix it. There
are test cases for checking the movement directions so when you are done try
running your code against the test classes.

\subsection{Up and down}

As you have seen, the circle that is controlled by the keyboard only has
left and right movement setup. Add two functions to model.cpp,
\functionname{move\_circle\_up} and \functionname{move\_circle\_down} that
add vertical movement to the keyboard controlled circle.

\subsection{Click not Hover}

Currently, the position of the mouse is always updating the position of its
corresponding circle and the color of these circles then changes based on
the current position of the mouse. However, we want the game to only update
the position of mouse circle when we click. You have seen an example of
detecting mouse click events in the previous class materials so consult those
for format and functions to use for detecting a click and update the logic for
when your program decides to change the color of the circles.

\subsection{Testing}

The current test file has a few example test functions that should run for
your code. Go ahead and add two test functions modeled after
\functionname{move\_circle\_left} and \functionname{move\_circle\_right} to
verify that the up and down movement code you added works. The test
\functionname{move\_get\_state} checks that the state will sometimes be
touching, fill in the alternative test for state
\functionname{move\_get\_not\_state} to check that the state will sometimes be
not\_touching.

\end{document}
