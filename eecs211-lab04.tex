\RequirePackage{fixltx2e}
\documentclass{tufte-handout}

% Be sure to check and update this:
\def\cLionurl{https://www.jetbrains.com/clion/download}
\def\macSDLurl{https://users.eecs.northwestern.edu/~jesse/course/eecs211/files/SDL2-all.dmg}
\def\MinGWSDLurl{https://users.eecs.northwestern.edu/~jesse/course/eecs211/files/MinGW-SDL2.exe}
\def\appleclangurl{https://developer.apple.com/downloads/}
\def\jetbrainsAccounturl{https://www.jetbrains.com/shop/eform/students}

\def\ThisLabBase{eecs211-lab04}
\def\ThisLabUrl{\LabBaseUrl/\ThisLabBase.zip}

\usepackage{eecs211-lab}
\title{EECS 211 Lab 4}
\author{C++ Toolchain Setup}
\date{Winter 2019}

\begin{document}

\maketitle

Today we begin programming in C++ and the GE211 game engine in a minimal
example game. The game is quite simple: You control two circles on the
screen---one with the mouse and one with the keyboard---and when the two
circles overlap, one changes color. As you will see, however, it comes
with a bug.

Before we can get started we need to install our C++ and GE211
development environment. You'll need a C++ compiler, the CLion IDE, and
the SDL2 graphics libraries. Read on\ldots

\section{Software registration \& downloads}

First, register\marginnote{\url{\jetbrainsAccounturl}} for a student
JetBrains account on \href{\jetbrainsAccounturl}{their website}.
You will receive an email that you need later in this process.

For all platforms, you will need to download the
\href{\cLionurl}{CLion installer}\marginnote{\url{\cLionurl}}.
Additional downloads vary by platform:

\begin{description}
  \item[Windows]
    You will need to\marginnote{\url{\MinGWSDLurl}}
    download our \href{\MinGWSDLurl}{custom installer} for
    \progname{MinGW-w64} with SDL2.
    This comes with C and C++ compilers as well as the SDL2 graphics
    library.
  \item[Mac]
    You will need to\marginnote{\url{\macSDLurl}}
    download our custom \href{\macSDLurl}{disk image}
    containing the SDL2 graphics library.
  \item[Linux]
    Make sure you have a working C++14 toolchain installed. You should
    also install the development packages for \filename{SDL2},
    \filename{SDL2\_image}, \filename{SDL2\_ttf}, and
    \filename{SDL2\_mixer}.
\end{description}

\section{Toolchain setup}

\subsection{Windows}

On Windows, you need to install \progname{MinGW-w64} (the C++ compiler):

\begin{enumerate}

  \item Run the \filename{MinGW-SDL2.exe} installer and follow the
    prompts to install \progname{MinGW-w64}. You should usually install it to
    \filename{C:\textbackslash MinGW}, but wherever you install it, take
    note, as you will have to configure CLion to find it later.

  \item Follow the instructions in your JetBrains registration email to
    activate your account.

\item Run the CLion installer. Most defaults should be fine, but you
  should check all of the ``Create associations'' boxes when they
    appear.

   Set the toolchain in CLion to the location where you installed
    \progname{MinGW}. The folder you select should contain subfolders
    with names like \filename{bin} and \filename{lib}. Ignore any
    warnings about version numbers.

\end{enumerate}

\subsection{Mac}

Mac OS automatically installs its toolchain when you attempt to use it
from the command line for the first time; you will still have to
install the SDL2 libraries yourself.

\begin{enumerate}

  \item Thus, to install developer tools, run the \textsf{Terminal.app}
    program (from \textsf{/Applications/Utilities}) to get a command
    prompt. At the prompt, type
\begin{CmdLine}
\prompt clang
\end{CmdLine}
and press return. If it prints \texttt{clang:~error:~no input files}
then you have it installed already. Otherwise, a dialog box will pop
up and offer to install the command-line developer tools for you. Say yes.

(Alternatively, you can install the latest version of Command Line Tools
for OS X manually from \href{\appleclangurl}{Apple}%
\marginnote{\url{\appleclangurl}}, or install XCode from the App Store.)

  \item Once you have the developer tools installed, you need to install
    the SDL2 libraries. Open the \textsf{SDL2-all.dmg} disk image and
    drag the four frameworks into the linked
    \textsf{/Library/Frameworks} directory. You may have to authenticate
    as an administrator.

  \item Follow the instructions in your JetBrains registration email to
    activate your account.

  \item Run the CLion installer---defaults should be fine.

\end{enumerate}

\section{The game}

\subsection{Getting the starter code}

For this lab, the starter code is provided as a ZIP file here:
\url{\ThisLabUrl}. Extract the archive file into a directory in the
location of your choosing. Once you have your new directory containing
the starter files, you can open it in CLion.

Be careful, as CLion will only work correctly if you open the \emph{main
project directory} with the \filename{CMakeLists.txt} in it. If you open
any other directory, CLion may create a \filename{CMakeLists.txt} for
you, but it won't work properly.

\subsection{Inverted control}

Currently, there is an bug in this code. Run the program and try to
control the circle with your left and right arrow keys, and you will
move in the opposite direction of what you intend. Locate the code for
this---hint: it's in the model---and fix it.

There are test cases for checking the model's movement, so when you are
done try running your code against the tests.

\subsection{Up and down}

As you have seen, the circle that is controlled by the keyboard only
moves horizontally right now.
Add two member functions to the \verb!Model! struct,
\verb!Model::move_large_circle_up()! and
\verb!Model::move_large_circle_down()!, and connect them to the keyboard
by modifying the \verb!Game::on_key(Key)! member function in
\filename{game.cpp}.

\subsection{Click, not hover}

Currently, the position of the smaller circle tracks the position of the
mouse. However, what if we want the game to only update the position of
smaller circle when we click? To detect mouse clicks, you will have to
override the \verb!Abstract_game::on_mouse_down(Mouse_button, Position)!
function in the \verb!Game! struct. See the documentation
\href{https://tov.github.io/ge211/classge211_1_1_abstract__game.html\#a6d88b5777c0a08fe261bc39c0694dd4f}{here}.

\subsection{Testing}

The current tests include a few examples that should pass
for your code. Add two new test cases for the new functions you added to
the model, and verify that they do what you expect.

There is a test case that checks that the state will sometimes be
\verb!Collision_state::touching!. Fill in the final test for checking
when the circles aren't touching.

\end{document}
