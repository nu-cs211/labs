%\documentclass[11pt]{article}
\documentclass{tufte-handout}

%\usepackage{microtype}
%\usepackage{mathspec}
%\usepackage{fullpage}
%\usepackage[dvipsnames]{xcolor}
%\usepackage{titlesec}
%\usepackage{multicol}
%\usepackage{parskip}
%\usepackage{211base}
\usepackage{211lab}

\LabInfo{4}{Debugging}

%\setallmainfonts{Charter}
%\setallsansfonts{Latin Modern Sans}
%\setallmonofonts[Scale=MatchLowercase]{Menlo}

%\newcommand\keyterm[1]{\textcolor[HTML]{006600}{\emph{#1}}}

\newcommand\dbc[2]{{%
  \sffamily\upshape\mdseries
  \textbf{{#1}}%
  #2}}

%\newcommand\abw[1]{$\langle\mathit{#1}\rangle$}
%
%\newcommand\keycombo[1]{{\sffamily\upshape\bfseries#1}}

\newenvironment{commands}
{\bgroup
  \let\olditem\item
  \def\item[##1]{\olditem[\keycombo{##1}]}%
  \begin{description}}
  {\end{description}\egroup}

%\titleformat{\section}
%  {\normalfont\sffamily\LARGE\mdseries}
%  {\thesection}{0.6em}{}
%\titleformat{\subsection}
%  {\normalfont\sffamily\large\mdseries}
%  {\thesubsection}{0.8em}{}
%
%\pagestyle{empty}
%\definecolor{Abbreviation}{HTML}{000000}
%
%\setlength{\columnsep}{1cm}
%\raggedright

\begin{document}

%\begin{center}
%  \sffamily
%  \COURSENO\ – \TERM\par
%  \Huge
%  Debugging
%  \Large
%  with GDB \& Emacs
%\end{center}

\maketitle

Today we are going to practice debugging using \progname{emacs} and
\progname{GDB}.

\progname{GDB} is a debugger lets you see inside your program while it runs. You can step
through it line by line, or you can choose a particular place to stop—a
\textbf{breakpoint}—and then have it continue at full speed until it reaches
the point where you’ve asked it to pause. You can print out the values of
objects, modify their values, and even call functions.


\section{Getting Started}

The starter code is available at \filename{\ThisLabTgz}, so you can
extract it into your current working directory with the command

\begin{CmdLine*}
  \C tar -xkvf \ThisLabTgz\\
\end{CmdLine*}

\noindent Your current working directory should now contain a
subdirectory called \filename{\ThisLabBase}.


\subsection{Revised Setup}

We've also improved several things about the 211 environment setup
to solve a couple of common student mistakes and ensure it doesn't
mess with future classes you take. You should re-run the installer now:

\begin{CmdLine*}
  \C \plaintilde cs211/setup211\\
\end{CmdLine*}

After you have done so, log out and back in.

Now, whenever you log in, instead of running \progname{fish}, you
will run a new command \progname{211}, which will set up the environment
and execute \progname{fish} in a single step:

\begin{CmdLine*}
  \C 211\\
\end{CmdLine*}

We'll remind you each time you log in with a message that says: ``Run ‘211’ for
the CS 211 development environment''.


\subsection{Guide to Emacs and GDB}

For this lab, we will provide you with some code that needs to be debugged.
Afterwards, hopefully you see some value in using \progname{GDB} on your own
code when you are having problems.

To use the debugger, you need to compile your C code with the \varname{-g}
flag. The \filename{Makefile}s we supply you with in each lab and homework
comes with that enabled already, so you should be all set. But if you run
\progname{cc} by hand, or use a future project with a different build system,
don't forget to pass it the \varname{-g} flag.

\progname{GDB} can be run directly from the shell, but if you run it in Emacs then Emacs
will help you out by showing you the debugger’s position in your code. You
\emph{definitely} want that, because the way \progname{GDB} shows context on its own is
not very friendly. But to use \progname{GDB} from Emacs, you’ll need to learn a bit more
Emacs. In particular, you need to understand windows and buffers, and you need
to be able to issue Emacs “\keycombo{M-x}” commands.

\section{Emacs Keys \& Commands}

\subsection{Windows \& Buffers}

Emacs lets you edit more than one file at a time, and that can include
  files that are open but aren’t currently displayed. Each thing you’re
  working on (usually a file, but possibly a compiler run or \progname{GDB}
  session), whether visible or not, is kept in a \textbf{buffer}. The
  rectangles in the terminal that some buffers are displayed in are called
  \textbf{windows}. You can open and close windows, switch between them,
  and switch which buffer is displayed in each window.

\begin{commands}
\item[C-x 2] Split this window in 2.
\item[C-x 1] Close all windows but this 1.
\item[C-x 0] Close the current window. (That's a zero.)
\item[C-x o] Other window, go there.
\item[C-x b] View a different buffer in this window.
\item[C-x C-b] Show a list of buffers.
\item[C-x C-f] Find a file to open in this window.
\end{commands}

\subsection{Running Things}

\keycombo{M-x} introduces commands that are whole words (or Emacs Lisp
  function names) rather than just keystrokes. So when you press it, a
  command line (called the ``minibuffer'') will open at the bottom of
  Emacs for you to type the rest of the command (and press Enter).

  Ah, but how do you type \textsf{M(eta)-}, a modifier key that your
  keyboard likely doesn’t have? You can configure your terminal to send
  a key you do have as \textsf{Meta}—usually you’d map \textsf{Alt} on
  Windows or \textsf{Option} on Mac. But if you haven’t set that up, you
  can always type an \keycombo{M-k} combination by typing
  \textsf{Escape} followed by the particular $k$.

  So to type \keycombo{M-x}, you can press \textsf{Escape} followed by
  \textsf{x}, regardless of your particular hardware or configuration.
  \keycombo{Option-x} or \keycombo{Alt-x} can be made to work and might
  already.

\begin{commands}

\item[M-x compile] Run the compiler. (Emacs will ask what command to
  run. If you’re currently editing a file in \filename{src/} or
    \filename{test/} then \progname{make} won’t find the \filename{Makefile} in there; change the
    compilation command to \cmdline{cd .. \&\& make -k} to \progname{cd} up a
    level before running \progname{make}.)

\item[M-x recompile] Run the compiler again with the same command.

\item[M-x shell] Open a shell in a buffer.

\item[M-x gdb] Start \progname{GDB}. (The default command that Emacs suggests is
  good.)

\end{commands}

\subsection{Getting Unstuck}

At some point, you may find yourself stuck in the minibuffer with Emacs
complaining about every keystroke. To get out, use:

\begin{commands}
\item[C-g] Quit the current operation.
\end{commands}

\subsection{Help}

\begin{commands}
\item[C-h a] Search for commands \emph{\underline  apropos to} a word.
\item[C-h d] Search the \underline  documentation.
\item[C-h k] Get help on a \underline  key combination.
\item[C-h l] See your \underline  last 300 keystrokes (to find
  out what you did just now).
\item[C-h ?] Get help on help.
\end{commands}

\section{GDB Commands}

\begin{description}

  \item[\dbc{h}{elp}] Displays help; follow it with a command name to
    get help on that command.

  \item[$\mathit{Enter}$] Repeats the previous command
    again.

\end{description}

\subsection{Finding \& Loading}

\begin{description}
  \item[\dbc{file}{} \metavar{FILE}] Tells GDB where to load your program
    from. This is relative to GDB’s working directory, so
    something like \verb!../count!.

  \item[\dbc{pwd}{}] Prints GDB’s working directory.

  \item[\dbc{cd}{} \metavar{DIR}] Changes GDB’s working directory.
\end{description}

\subsection{Starting \& Stopping}
\begin{description}

  \item[\dbc{b}{reak} \metavar{point}] Sets a breakpoint at a function, a
    line number, or \metavar{file}:\metavar{line}).

  \item[\dbc{clear}{} \metavar{point}] Deletes any breakpoints at
    \metavar{point}; omit \metavar{point} to clear a breakpoint on the current
    line.

  \item[\dbc{r}{un} \metavar{args}\ldots] Runs your program from the start;
    optionally passes \metavar{args} as command-line arguments.

  \item[\keycombo{C-c C-c}] Pauses the running program. Code will stop on
    whatever line was executing when the key combination was pressed.

  \item[\dbc{n}{ext}] Executes until the next line of your program, not
    looking inside function calls.

  \item[\dbc{s}{tep}] Executes one small step of your program, including
    stepping into function calls.

  \item[\dbc{fin}{ish}] Resumes your program for the remainder of the
    current function call.

  \item[\dbc{c}{ontinue}] Resumes your program from where it’s paused
    and runs to the next breakpoint.

\end{description}

\subsection{Peeking \& Poking}
\begin{description}

  \item[\dbc{p}{rint} \metavar{EXPR}] Prints the value of a variable in your
    program (in scope at the execution point), or the value of a larger
    expression in the context of your program. (Can even call
    functions!)

  \item[\dbc{set}{ variable} \metavar{VARNAME} = \metavar{EXPR}] Modifies the
    value of a variable in your program.

\end{description}

\vfill\eject
\section{Debugging some code}

First, enter the \filename{lab04/} directory and run \progname{make}.

\subsection{Fixing an infinite loop}

The executable \progname{infinite} is created from \filename{src/infinite.c}
Try compiling the code and running it first. As you may have guessed, it runs
forever.
%
\marginnote{Remember that you can use the command \keycombo{C-c} in the shell to
terminate a running program.}
%
This isn't what it's supposed to do, however. It has a bug. Let's use GDB to
figure out where the unintentionally-infinite loop is.

\begin{enumerate}
  \item Open \progname{emacs} without specifying any filename.

  \item Start a \progname{GDB} session with \keycombo{M-x gdb}. Then hit
    $\mathit{Enter}$ to execute the command and open a multi-window layout. The
    top-left window is where you type commands. The top-right window displays
    local variables when code is paused. The middle-left window displays the code
    currently open.

  \item In the top-left window, type \keycombo{pwd} to see what directory
    \progname{GDB} is running in.

  \item In the top-left window, type \keycombo{file infinite} to start
    debugging the \progname{infinite} executable. Note that you have to specify
    the relative path to the executable you want to debug, so if \keycombo{pwd}
    didn't show \progname{GDB} running in \filename{lab04/} you will have to
    type the path to \progname{infinite} instead.

  \item In the top-left window, type \keycombo{run} to start running the code.
    Just like when you run it outside of \progname{GDB} this code will run
    forever without printing anything.

  \item Press the key combination \keycombo{C-c C-c} to pause the running code.
    The middle-left window will now display what line of code was running when
    execution was paused.

  \item In the top-left window,
    %
    \marginnote{You can hit $\mathit{Enter}$ in \progname{GDB} to run the
    previous command again. This is really useful when calling \keycombo{step}
    multiple times in a row!}
    %
    type \keycombo{step} to move through code line-by-line. As you do, you will
    see the current line changing in the middle-left window and the local
    variables and their values changing in the top-right window.

  \item You can also use the \keycombo{print} command to display the value of
    variables. For example, you can type \keycombo{print s} to display the
    input argument to the function.

  \item Step through the code for a while, watching the local variable values,
    and see if you can figure out why this code runs infinitely.

  \item To close the \progname{GDB} session, you can type \keycombo{quit} in
    the top-left window and then press \keycombo{C-x C-c} to exit
    \progname{emacs}.

\end{enumerate}


%1. this code runs forever. Let's figure out where the unintentionally-infinite loop is
% - run code through GDB
% - break
% - determine what function you are in
%
%2. now figure out what the error is
% - print variables
% - step through code printing variables


\subsection{Understanding nonfunctional code}

Broken code doesn't always infinite loop. Sometimes it completes quickly, but
doesn't provide a correct result. Let's use \progname{GDB} to take a look at a
second program, \progname{broken}, that runs but doesn't work properly.

First, try running \progname{broken} in the shell and take a look at the code
for it in \filename{src/broken.c}. It currently prints out the wrong summation
for the array. It does finish running though, so this time we'll have to debug
it using \textit{breakpoints}.

\begin{enumerate}
  \item The first few steps are the same as last time: open \progname{emacs},
    start a \progname{GDB} session, and use the \keycombo{file} command to
    specify which executable you are debugging (this time it should be
    \progname{broken}).

  \item In the top-left window, type \keycombo{break sum_array}. This will create
    a breakpoint at the start of that function, any time it is called.

  \item In the top-left window, type \keycombo{run} to start running the code.
    This time, execution will pause automatically once the breakpoint is reached.

  \item Like last time, use \keycombo{step} and the local variable values in the
    top-right window to determine what is going wrong and fix it.


\end{enumerate}


%1. this code returns immediately, but doesn't actually do what it claims to. We could stare at it a long time and try to figure out why, or we could use GDB to inspect it while it is running


\end{document}
