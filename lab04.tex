\documentclass[11pt]{article}

\usepackage{microtype}
\usepackage{mathspec}
\usepackage{fullpage}
\usepackage[dvipsnames]{xcolor}
\usepackage{titlesec}
\usepackage{multicol}
\usepackage{parskip}

\setallmainfonts{Charter}
\setallsansfonts{Latin Modern Sans}
\setallmonofonts[Scale=MatchLowercase]{Menlo}

\newcommand\keyterm[1]{\textcolor[HTML]{006600}{\emph{#1}}}

\def\LabTGZ{\$LAB211/lab04.tgz}

\newcommand\dbc[2]{{%
  \sffamily\upshape\mdseries
  \textbf{\textcolor{Abbreviation} {#1}}%
  #2}}
\newcommand\abw[1]{$\langle\mathit{#1}\rangle$}

\newcommand\keycombo[1]{{\sffamily\upshape\bfseries#1}}

\newenvironment{commands}
{\bgroup
  \let\olditem\item
  \def\item[##1]{\olditem[\keycombo{##1}]}%
  \begin{description}}
  {\end{description}\egroup}

\titleformat{\section}
  {\normalfont\sffamily\LARGE\mdseries}
  {\thesection}{0.6em}{}
\titleformat{\subsection}
  {\normalfont\sffamily\large\mdseries}
  {\thesubsection}{0.8em}{}

\pagestyle{empty}
\definecolor{Abbreviation}{HTML}{000000}

\setlength{\columnsep}{1cm}
\raggedright

\begin{document}%

\begin{center}
  \sffamily
  CS 211 – Spring 2020
  \par
  \Huge
  Debugging
  \Large
  with GDB \& Emacs
\end{center}

\begin{multicols}{2}

GDB is a debugger lets you see inside your program while it runs. You
  can step through it line by line, or you can choose a particular place
  to stop—a \keyterm{breakpoint}—and then have it continue at full speed
  until it reaches the point where you’ve asked it to pause. You can
  print out the values of objects, modify their values, and even call
  functions.

  A few things before we get started:

  % \begin{itemize}

      1) You’ll need some code to debug. You might try your current
      homework or previous one, or there’s some code from lecture
      involving strings and linked lists, prepared for this lab, that
      you can download from \textsf{\LabTGZ} (using \verb!curl! and
      \verb!tar zxvk! like usual).

      2) To use the debugger, you need to compile your C code with the
      \verb!-g! flag. The \verb!Makefile!'s we supply you with usually
      come with that already, but the \verb!Makefile! in
      \textsf{lab04.tgz} has it commented out, like so:

      \begin{verbatim}
      # CFLAGS    += -g\end{verbatim}

      You need to remove the \verb!#! at the beginning of the line to
      enable the option.

      3) GDB can be run directly from the shell, but if you run it
      in Emacs then Emacs will help you out by showing you the
      debugger’s position in your code. You \emph{definitely} want that,
      because the way GDB shows context on its own
      is not very friendly. But to use GDB from Emacs, you’ll need to
      learn a bit more Emacs. In particular, you need to understand
      windows and buffers, and you need to be able to issue Emacs
      “\textsf{M-x}” commands.

  % \end{itemize}

\section*{Emacs Keys \& Commands}

\subsection*{Windows \& Buffers}

Emacs lets you edit more than one file at a time, and that can include
  files that are open but aren’t currently displayed. Each thing you’re
  working on (usually a file, but possibly a compiler run or GDB
  session), whether visible or not, is kept in a \keyterm{buffer}. The
  rectangles in the terminal that some buffers are displayed in are called
  \keyterm{windows}. You can open and close windows, switch between them,
  and switch which buffer is displayed in each window.

\begin{commands}
\item[C-x 2] Split this window in 2.
\item[C-x 1] Close all windows but this 1.
\item[C-x 0] Close the current window. (That's a zero.)
\item[C-x o] Other window, go there.
\item[C-x b] View a different buffer in this window.
\item[C-x C-b] Show a list of buffers.
\item[C-x C-f] Find a file to open in this window.
\end{commands}

\subsection*{Running Things}

\textsf{M-x} introduces commands that are whole words (or Emacs Lisp
  function names) rather than just keystrokes. So when you press it, a
  command line (called the ``minibuffer'') will open at the bottom of
  Emacs for you to type the rest of the command (and press Enter).

  Ah, but how do you type \textsf{M(eta)-}, a modifier key that your
  keyboard likely doesn’t have? You can configure your terminal to send
  a key you do have as \textsf{Meta}—usually you’d map \textsf{Alt} on
  Windows or \textsf{Option} on Mac. But if you haven’t set that up, you
  can always type an \textsf{M-}$k$ combination by typing
  \textsf{Escape} followed by the particular $k$.

\begin{commands}

\item[M-x compile] Run the compiler. (Emacs will ask what command to run.
  Change it to \verb!cd .. && make -k! so that Make can find the
  Makefile.)

\item[M-x recompile] Run the compiler again with the same command.

\item[M-x shell] Open a shell in a buffer.

\item[M-x gdb] Start GDB. (The default command that Emacs suggests is
  good.)

\end{commands}

\subsection*{Getting Unstuck}

At some point, you may find yourself stuck in the minibuffer with Emacs
complaining about every keystroke. To get out, use:

\begin{commands}
\item[C-g] Quit the current operation.
\end{commands}

\subsection*{Help}

\begin{commands}
\item[C-h a] Search for commands \emph{\underline  apropos to} a word.
\item[C-h d] Search the \underline  documentation.
\item[C-h k] Get help on a \underline  key combination.
\item[C-h l] See your \underline  last 300 keystrokes (to find
  out what you did just now).
\item[C-h ?] Get help on help.
\end{commands}

\section*{GDB Commands}

\begin{description}

  \item[\dbc{h}{elp}] Displays help; follow it with a command name to
    get help on that command.

  \item[$\mathit{Enter}$] Repeats the previous command
    again.

\end{description}

\subsection*{Finding \& Loading}

\begin{description}
  \item[\dbc{file}{} \abw{FILE}] Tells GDB where to load your program
    from. This is relative to GDB’s working directory, so
    something like \verb!../count!.

  \item[\dbc{pwd}{}] Prints GDB’s working directory.

  \item[\dbc{cd}{} \abw{DIR}] Changes GDB’s working directory.
\end{description}

\subsection*{Starting \& Stopping}
\begin{description}

  \item[\dbc{b}{reak} \abw{point}] Sets a breakpoint at a function, a
    line number, or \abw{file}:\abw{line}).

  \item[\dbc{clear}{} \abw{point}] Deletes any breakpoints at
    \abw{point}; omit \abw{point} to clear a breakpoint on the current
    line.

  \item[\dbc{r}{un} \abw{args}\ldots] Runs your program from the start;
    optionally passes \abw{args} as command-line arguments.

  \item[\dbc{n}{ext}] Executes until the next line of your program, not
    looking inside function calls.

  \item[\dbc{s}{tep}] Executes one small step of your program, including
    stepping into function calls.

  \item[\dbc{c}{ontinue}] Resumes your program from where it’s paused
    and runs to the next breakpoint.

  \item[\dbc{fin}{ish}] Resumes your program for the remainder of the
    current function call.

\end{description}

\subsection*{Peeking \& Poking}
\begin{description}

  \item[\dbc{p}{rint} \abw{EXPR}] Prints the value of a variable in your
    program (in scope at the execution point), or the value of a larger
    expression in the context of your program. (Can even call
    functions!)

  \item[\dbc{set}{ variable} \abw{VARNAME} = \abw{EXPR}] Modifies the
    value of a variable in your program.

\end{description}

\end{multicols}

\end{document}
