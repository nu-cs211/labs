\RequirePackage{fixltx2e}
\documentclass{tufte-handout}

\def\ThisLabBase{eecs211-lab06}
\def\ThisLabUrl{\LabBaseUrl/\ThisLabBase.zip}

\usepackage{eecs211-lab}
\title{EECS 211 Lab 6}
\author{Bejeweled}
\date{Winter 2019}

\begin{document}

\maketitle

Today we will be looking at yet another a C++ program using the GE211 game
engine in a reasonably advanced example game. You may have played this game
in various forms such as in Candy Crush or other tile swapping games, but the
basic concept is to destroy sets of like tiles by swapping two tiles to
create a set. In this version of the game a set will be considered 3 or more
tiles of the same type. You can only make swaps that will result in the
destruction of a set.

This game uses the model--view--controller pattern not-yet-described in
class, which allows defining the look, user interaction, and ``business
logic'' of an interactive program as separate components. Provided are
the \texttt{Model} class which defines the internal game state, the
\texttt{View} class for rendering game to the screen, and the
\texttt{Controller} class for reacting to user input and tying it all
together. In addition, because the tiles in this game are rather complex
themselves, we have \filename{tiles.h} which sets up \texttt{Board\_Position}
(where something is on a board, basic operators for that position, and
functions for finding what is around it), the \texttt{Tile\_Data} (uses board
positions), defines a \texttt{Tile\_Handler} for processing special types of
tiles, the \texttt{Tile} struct (a combination of tile data and handlers with
the function for swapping with another tile), and finally a
\texttt{Tile\_Handler\_Reference} that points to the specific handler to be
used in a given tile. All of this allows for the creation of more tile handlers
which can each have special destructive powers. You can see that we have provided
the normal handler (just delete the set of tiles we created) and a horizontal
lazer (deletes all tiles in the row in addition to the set we created by swapping)
inside the \filename{tile\_handlers.cpp}.

\section{Getting the starter code}

For this lab, the starter code is provided as a ZIP file here:
\url{\ThisLabUrl}. Extract the archive file into a directory in the
location of your choosing. Once you have your new directory containing
the starter files, you can open it in CLion.
\marginpar{ Be careful, as CLion will only work correctly if you open
the \emph{main project directory} with the \filename{CMakeLists.txt} in
it. If you open any other directory, CLion may create a
\filename{CMakeLists.txt} for you, but it won't work properly.}


\section{General idea}

The version of Bejeweled that you have been given is not following all
of the rules we discussed earlier but is otherwise a fully functioning
version of Bejeweled. This game loads a board (defaults to 10 by 8) with
several (defaults to 6) groups - which will behave as tile colors for
grouping same colors - and as many types as you have tile handlers (starts
at 2). From here, the controller decides when to update a frame and in each
update \verb!model_.run_step()! is called which is the brains behind
finding what needs to change, detecting the set of tiles to be destroyed
(including any caused by destroying a special type), and removes them as
needed. Looking through this function and the other functions it uses in
the \verb^Model^ class should help you understand how the game is utilizing
tiles. The \verb^Controller^ also utilizes some of the \verb^Model^ functions
in \verb!Controller::on_mouse_up! which (when the view isn't going through
animations) on first click of a valid tile will select that tile and on
second click of another tile will attempt to swap them. Upon swapping and
creating a set to be destroyed, that set of tiles will be removed and the
tiles above them will be shifted down, and new random tiles will also be
shifted down to fill the gaps created at the top. All of these changes are
animated by the \verb^View^ class which will make the program unresponsive to
input while it displays the changes slow enough for you to actually see what
happens.

\section{More Valid Swaps}

We mentioned above that this version of Bejeweled does not follow all of the
rules we described at the beginning. Specifically, this version of Bejeweled
allows for any two tiles that are next to each other to be swapped instead of
limiting the swaps to only those that will create valid sets for destruction.

In \filename{controller.cpp} the function discussed above -
\verb!Controller::on_mouse_up! - uses \verb!model_.is_valid(bp)! to check that
the selection is inside the board. However, we want something more complicated
than just checking that the selection is on the board, we also want to make
sure that the selection has a valid swap. Your job is to modify the code in the
\verb^Model^ class to only tell the \verb^Controller^ a selection is valid if
it will also be valid for a swap. To accomplish this you will have tom

\begin{enumerate}
  \item Modify \verb!Model::is_valid! to only return true when the position is
    on the board and if the there is a possible valid swap. Remember, there is
    a function \verb!Model::is_valid_swap! that may be useful. As well, keep in
    mind that there are four possible ways that a tile could swap and the one
    you select only needs to be able to swap with one of them (although more
    won't hurt).

  \item Modify \verb!Model::is_valid_swap! to only return true when both tiles
    are on the board, next to each other, and will result in a set of tiles to
    be destroyed. Consult the aforementioned \verb!model_.run_step()! to see how
    valid groups for destruction are created - even bigger hint, look at
    \verb!Model::get_group_! to see how sets of a group of tiles are found.
\end{enumerate}

\section{Cooler Tile Handlers}

Currently, the special tile handlers are relatively boring. Your job is to add
a few more handlers to make the game more interesting. To add a tile handler
you will have to:

\begin{enumerate}
  \item Create a new class in \filename{tile\_handlers.h} following the style
    of Normal and Horizontal Lazer with your new name instead.

  \item Define the \verb!process_removal! function for your new class (follow
    the style of Normal and Horizontal Lazer here as well but make the inside
    of the function only choose the tiles to delete that you want) inside
    \filename{tile\_handlers.cpp}.
\end{enumerate}

Go ahead and add vertical lazer, destroy all tiles in this group, and
destroy all tiles on the diagonals (X) handlers. You will also need to modify
\filename{game.cpp} to know about these new types:

\begin{enumerate}
    \item Update \verb!types_count! to the correct number of types you've added.

    \item Inside the \verb!main()! function the tile handlers are instantiated
      and used in the call to \verb!Controller(...)! by adding a
      \verb!Tile_Handler_Reference! to the set of handlers the game knows about.
      Instantiate your own handlers and add the references to the game.
\end{enumerate}

\section{Testing}

Test destroy?

\end{document}
