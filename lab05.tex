\documentclass{tufte-handout}

\usepackage{211lab}
\LabInfo{5}{Welcome to C++}

% Be sure to check and update this:
\def\cLionurl{https://www.jetbrains.com/clion/download}
\def\macSDLurl{https://users.eecs.northwestern.edu/~jesse/course/eecs211/files/SDL2-all.dmg}
\def\MinGWSDLurl{https://users.eecs.northwestern.edu/~jesse/course/eecs211/files/MinGW-SDL2.exe}
\def\appleclangurl{https://developer.apple.com/downloads/}
\def\jetbrainsAccounturl{https://www.jetbrains.com/shop/eform/students}

\begin{document}

\maketitle

Today we begin programming in C++ and the GE211 game engine in a minimal
example game. The game is quite simple: You control two circles on the
screen---one with the mouse and one with the keyboard---and when the two
circles overlap, one changes color. As you will see, however, it comes
with a bug.

Before we can get started we need to install our C++ and GE211
development environment. You'll need a C++ compiler, the CLion IDE, and
the SDL2 graphics libraries. Read on\ldots

\section{The game}

\subsection{Getting the starter code}

For this lab, the starter code is provided as a ZIP file here:
\url{\ThisLabUrl}. Extract the archive file into a directory in the
location of your choosing. Once you have your new directory containing
the starter files, you can open it in CLion.

Be careful, as CLion will only work correctly if you open the \emph{main
project directory} with the \filename{CMakeLists.txt} in it. If you open
any other directory, CLion may create a \filename{CMakeLists.txt} for
you, but it won't work properly.

\subsection{Inverted control}

Currently, there is an bug in this code. Run the program and try to
control the circle with your left and right arrow keys, and you will
move in the opposite direction of what you intend. Locate the code for
this---hint: it's in the model---and fix it.

There are test cases for checking the model's movement, so when you are
done try running your code against the tests.

\subsection{Up and down}

As you have seen, the circle that is controlled by the keyboard only
moves horizontally right now.
Add two member functions to the \verb!Model! struct,
\verb!Model::move_large_circle_up()! and
\verb!Model::move_large_circle_down()!, and connect them to the keyboard
by modifying the \verb!Game::on_key(Key)! member function in
\filename{game.cpp}.

\subsection{Click, not hover}

Currently, the position of the smaller circle tracks the position of the
mouse. However, what if we want the game to only update the position of
smaller circle when we click? To detect mouse clicks, you will have to
override the \verb!Abstract_game::on_mouse_down(Mouse_button, Position)!
function in the \verb!Game! struct. See the documentation
\href{https://tov.github.io/ge211/classge211_1_1_abstract__game.html\#a6d88b5777c0a08fe261bc39c0694dd4f}{here}.

\subsection{Testing}

The current tests include a few examples that should pass
for your code. Add two new test cases for the new functions you added to
the model, and verify that they do what you expect.

There is a test case that checks that the state will sometimes be
\verb!Collision_state::touching!. Fill in the final test for checking
when the circles aren't touching.

\end{document}
